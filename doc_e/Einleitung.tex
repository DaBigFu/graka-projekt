\chapter{Einleitung}
\label{cha:Einleitung}

An der Hochschule Aalen besteht ein breites Spektrum an Arbeiten welche sich mit Bild- und Videobearbeitung beschäftigen oder einer Monitorausgabe für andere Zwecke verwenden. Jedoch nutzen all dieses Projekte bestehende VGA-Schnittstellen mit einer maximalen Farbauflösung von 8-Bit, was eine Darstellung von$\ 2^{8}=256$ Farben ermöglicht. Allerdings ist dies relativ wenig im Vergleich zum heutigen Standard von 24-Bit, also$\ 2^{24}=16777216$ Farben.

Des Weiteren sind derartige Darstellungsformate mit einem Datenaufwand verbunden, welcher die Kapazitäten eines Cyclone 3 übersteigt. Also muss auch eine Anbindung an einen externen Speicher entwickelt werden. Hierfür wurde der auf dem DE0-Board verbaute SDRAM verwendet.

Ziel dieser Arbeit ist es eine Grafikkarte mit einer Farbauflösung von 24-Bit zu entwickeln und diese in ein bestehendes Projekt einzuarbeiten. Hierfür wurde das Projekt von Herrn Christoph Paa verwendet, welcher zeitgleich ein Programm zur Bildbearbeitung auf Basis eines FPGA im Rahmen seiner Projektarbeit entwickelte.