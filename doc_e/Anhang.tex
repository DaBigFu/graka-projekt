\chapter{Anhang}
\label{cha:Anhang}

\section*{Inhalt der CD-ROM}

\begin{table}[h!]
\setlength{\tabcolsep}{1ex}
\def\arraystretch{1.20}
\setlength{\tabcolsep}{1ex}
\small
\begin{tabular}{p{5cm}l}
\multicolumn{2}{l}{\textbf{/Datenblätter}}\\
DE0\_Schaltplan.pdf & Schaltplan des Terasic DE0 Boards \\ 
DE0\_User\_manual.pdf & DE0 Datenblatt \\
Zentel\_SDRAM.pdf & Zentel SDRAM Datenblatt\\
\end{tabular} 
\end{table}

\begin{table}[h!]
\setlength{\tabcolsep}{1ex}
\def\arraystretch{1.20}
\setlength{\tabcolsep}{1ex}
\small
\begin{tabular}{p{5cm}l}
\multicolumn{2}{l}{\textbf{/Dokumentation}}\\
Dokumentation.tex & Latex Basis-File \\ 
Dokumentation.pdf & Dokumentation im .pdf-Format \\
/images/... & Unterordner für die in der Dokumentation verwendeten Bilder\\
\end{tabular} 
\end{table}

\begin{table}[h!]
\setlength{\tabcolsep}{1ex}
\def\arraystretch{1.20}
\setlength{\tabcolsep}{1ex}
\small
\begin{tabular}{p{5cm}l}
\multicolumn{2}{l}{\textbf{/Eagle-Files}}\\
VGA\_Erweiterung.brd & Board-Layout der Erweiterungsplatine \\ 
VGA\_Erweiterung.sch & Schaltplan des Erweiterungsboards \\
\end{tabular} 
\end{table}

\begin{table}[h!]
\setlength{\tabcolsep}{1ex}
\def\arraystretch{1.20}
\setlength{\tabcolsep}{1ex}
\small
\begin{tabular}{p{5cm}l}
\multicolumn{2}{l}{\textbf{/LTspiceSimulationen}}\\
4bit.asc & Simulation des DACs auf dem DE0 Board \\ 
8bit.asc & Simulation des entworfenen DACs \\
8bit\_optimal.asc & Entwurf für den Fall einheitlicher Vielfacher von R7\\
\end{tabular} 
\end{table}

\begin{table}[h!]
\setlength{\tabcolsep}{1ex}
\def\arraystretch{1.20}
\setlength{\tabcolsep}{1ex}
\small
\begin{tabular}{p{5cm}l}
\multicolumn{2}{l}{\textbf{/MatlabGUI}}\\
graka\_gui.m & Matlab GUI \\ 
\end{tabular} 
\end{table}

\begin{table}[h!]
\setlength{\tabcolsep}{1ex}
\def\arraystretch{1.20}
\setlength{\tabcolsep}{1ex}
\small
\begin{tabular}{p{5cm}l}
\multicolumn{2}{l}{\textbf{/LTspiceSimulationen}}\\
4bit.asc & Simulation des DACs auf dem DE0 Board \\ 
8bit.asc & Simulation des entworfenen DACs \\
8bit\_optimal.asc & Entwurf für den Fall einheitlicher Vielfacher von R7\\
\end{tabular} 
\end{table}

\begin{table}[h!]
\setlength{\tabcolsep}{1ex}
\def\arraystretch{1.20}
\setlength{\tabcolsep}{1ex}
\small
\begin{tabular}{p{5cm}l}
\multicolumn{2}{l}{\textbf{/Testbilder}}\\
futurama.bin & Futurama-Bitmap ohne header \\ 
futurama.bmp & Futurama-Bitmap \\
Schloss.bin & Schloss-Bitmap ohne header\\
Schloss.bmp & Schloss-Bitmap\\
\end{tabular} 
\end{table}

\begin{table}[h!]
\setlength{\tabcolsep}{1ex}
\def\arraystretch{1.20}
\setlength{\tabcolsep}{1ex}
\small
\begin{tabular}{p{5cm}l}
\multicolumn{2}{l}{\textbf{/VHDL-Code}}\\
cmd\_dec\_sdram\_cntrl.vhd & RS232-Komandodecoder und SDRAM-Controller\\
dbg\_decoder.vhd & Debug-Modul \\
graka\_pack.vhd & Funktionspackage \\
PLL.vhd & 166MHz PLL \\
PLL65.vhd & 65MHz PLL \\
RS232.vhd & RS232 Schnittstelle \\
single\_port\_ram.vhd & Single Port Ram Komponente \\
ss\_decoder.vhd & Decoder für Siebensegment Anzeige\\
VGA.vhd & VGA-Schnittstelle \\
VHDL\_graka.qpf & Quartus Projekt-File\\
\end{tabular} 
\end{table}